% Options for packages loaded elsewhere
\PassOptionsToPackage{unicode}{hyperref}
\PassOptionsToPackage{hyphens}{url}
%
\documentclass[
]{article}
\usepackage{amsmath,amssymb}
\usepackage{iftex}
\ifPDFTeX
  \usepackage[T1]{fontenc}
  \usepackage[utf8]{inputenc}
  \usepackage{textcomp} % provide euro and other symbols
\else % if luatex or xetex
  \usepackage{unicode-math} % this also loads fontspec
  \defaultfontfeatures{Scale=MatchLowercase}
  \defaultfontfeatures[\rmfamily]{Ligatures=TeX,Scale=1}
\fi
\usepackage{lmodern}
\ifPDFTeX\else
  % xetex/luatex font selection
\fi
% Use upquote if available, for straight quotes in verbatim environments
\IfFileExists{upquote.sty}{\usepackage{upquote}}{}
\IfFileExists{microtype.sty}{% use microtype if available
  \usepackage[]{microtype}
  \UseMicrotypeSet[protrusion]{basicmath} % disable protrusion for tt fonts
}{}
\makeatletter
\@ifundefined{KOMAClassName}{% if non-KOMA class
  \IfFileExists{parskip.sty}{%
    \usepackage{parskip}
  }{% else
    \setlength{\parindent}{0pt}
    \setlength{\parskip}{6pt plus 2pt minus 1pt}}
}{% if KOMA class
  \KOMAoptions{parskip=half}}
\makeatother
\usepackage{xcolor}
\usepackage[margin=1in]{geometry}
\usepackage{graphicx}
\makeatletter
\def\maxwidth{\ifdim\Gin@nat@width>\linewidth\linewidth\else\Gin@nat@width\fi}
\def\maxheight{\ifdim\Gin@nat@height>\textheight\textheight\else\Gin@nat@height\fi}
\makeatother
% Scale images if necessary, so that they will not overflow the page
% margins by default, and it is still possible to overwrite the defaults
% using explicit options in \includegraphics[width, height, ...]{}
\setkeys{Gin}{width=\maxwidth,height=\maxheight,keepaspectratio}
% Set default figure placement to htbp
\makeatletter
\def\fps@figure{htbp}
\makeatother
\setlength{\emergencystretch}{3em} % prevent overfull lines
\providecommand{\tightlist}{%
  \setlength{\itemsep}{0pt}\setlength{\parskip}{0pt}}
\setcounter{secnumdepth}{5}
\usepackage{booktabs}
\usepackage{caption}
\usepackage{longtable}
\usepackage{colortbl}
\usepackage{array}
\ifLuaTeX
  \usepackage{selnolig}  % disable illegal ligatures
\fi
\IfFileExists{bookmark.sty}{\usepackage{bookmark}}{\usepackage{hyperref}}
\IfFileExists{xurl.sty}{\usepackage{xurl}}{} % add URL line breaks if available
\urlstyle{same}
\hypersetup{
  pdftitle={Determining Core Words for Mandarin-Speaking Children},
  pdfauthor={Written by: Morgan Wood; Client: Hsiao-Ting Su},
  hidelinks,
  pdfcreator={LaTeX via pandoc}}

\title{Determining Core Words for Mandarin-Speaking Children}
\author{Written by: Morgan Wood \and Client: Hsiao-Ting Su}
\date{15 November, 2023}

\begin{document}
\maketitle
\begin{abstract}
This is my abstract paragraph 1. With details.

This is paragraph 2
\end{abstract}

\hypertarget{overview}{%
\section{Overview}\label{overview}}

Determining core words for a set of individuals is an important first
step in developing Aided Augmentative and Alternative Communication
devices (Aided AAC) tailored for those individuals. Aided AAC are either
electronic or non-electronic devices used as an alternative to
traditional verbal communication. Examples of Aided AAC are
communication boards or speech generating devices.

The goal of this report is to display a set of potential core words for
Mandarin speaking children between the ages of 4 and 6 to aid in the
creation of Aided AAC for these individuals.

To determine core words, multiple text files containing transcribed
stories told by children are deconstructed into a list of words. In our
setting, a word can by a single Chinese character or multiple Chinese
characters.

For this report, I consider four different groups of children each with
\(n\) children within the group and a total number of \(w\) words
spoken. The groups considered are listed below.

\begin{itemize}
\item
  All Children (Ages 4 through 6) (\(n\) = 127, \(w\) = 58773)
\item
  Children of Age 4 (\(n\) = 40, \(w\) = 17893)
\item
  Children of Age 5 (\(n\) = 50, \(w\) = 22787)
\item
  Children of Age 6 (\(n\) = 37, \(w\) = 18093)
\end{itemize}

For each group of children, two statistics are computed for each
distinct word spoken. First, if a word is spoken by \(c\cdot100\%\) of
children in the group, the word is assigned a commonality \(c\). The
frequency that a word is spoken is also looked at. From this we can
obtain a relative frequency \(r\) which will be the number of times per
1000 words the word is spoken. The total frequency of each word is also
reported.

Within this report, a word is considered core if at least 30\% of
children spoke the word and the word is spoken with a relative frequency
of at least 0.5 times per 1000 words. I refer to .3 as the commonality
threshold and .5 as the frequency threshold.

\hypertarget{choice-of-commonality-and-frequency-threshold}{%
\section{Choice of Commonality and Frequency
Threshold}\label{choice-of-commonality-and-frequency-threshold}}

I begin this report with a comment on the choice of threshold for
declaring a core word.

While commonality and frequency represent two different statistics of
the data, they are related. For example, if a word has commonality
\(1\), then this requires that each individual in the study spoke the
word. This implies that the word is spoke with a cumulative frequency at
least equal to the number of individuals \(n\). More generally, if a
word has a commonality of \(c\), the relative frequency must be at least

\[1000 \cdot\frac{c\cdot n}{w}\] where \(w\) is the total number of
words spoken.

In fact, if we require a commonality threshold of \(c=0.3\), this forces
a relative frequency \(r\) of at least the following for each group.

\begin{itemize}
\item
  All Children (Ages 4 through 6): \(r \geq\) 0.6482569
\item
  Children of Age 4: \(r \geq\) 0.6706533
\item
  Children of Age 5: \(r \geq\) 0.6582701
\item
  Children of Age 6: \(r \geq\) 0.6134969
\end{itemize}

Thus, unless a frequency threshold is above the number above, this
threshold will not play a role in choosing which words are considered
core. If it is desired for frequency to play a role in selection core
words, I suggest selecting a frequency threshold strictly larger than
the values listed above.

Regardless of this, for the remainder of this report I continue with the
commonality and frequency thresholds \(c=.3\) and \(r=.5\),
respectively, because this was requested. This implies that core words
were actually only chosen with the condition that the commonality was at
least 30\%.

\hypertarget{core-words}{%
\section{Core Words}\label{core-words}}

Below, I give a table that previews the first 6 core words for children
in the combined group. The core words listed below are in order of
decreasing frequency. In total, there were 122 core words identified.
The complete list of core words can be found on GitHub in the excel file
``core\_words\_processed.xlsx''\footnote{Link:
  \url{https://github.com/smithmor/Aided_AAC_Analysis/blob/7cab44371c38a43652a1a81eacb7862b4a6128c9/core_words_processed.xlsx}}
under the ``Core Words for Combined Group'' tab.

\begin{longtable}{l|rrr}
\caption*{
{\large Preview of Core Words Across All Ages}
} \\ 
\toprule
\multicolumn{1}{l}{} & Composite.frequency & Relative.frequency.per.1000.words & Commonality \\ 
\midrule\addlinespace[2.5pt]
我 & 2656 & 45.19 & 0.99 \\ 
的 & 2437 & 41.46 & 0.99 \\ 
個 & 1917 & 32.62 & 0.98 \\ 
是 & 1507 & 25.64 & 0.98 \\ 
有 & 1466 & 24.94 & 0.98 \\ 
就 & 1165 & 19.82 & 0.96 \\ 
\bottomrule
\end{longtable}

Similarly, core words for other groups can be found. Below is a preview
of core words for 4-year-olds, 5-year-olds, and 6-year-olds, which had a
total of 121, 125, and 126 core words identified, respectively. It is
worth noting that the number of core words increase with each age group.
This matches the intuition that older children will have a broader
vocabulary. Also of interest is the observation that one child in the
6-year-old group did not use the word ``我'' which corresponds to the
English word ``I''.

\begin{longtable}{l|rrr}
\caption*{
{\large Preview of Core Words for 4-Year-Olds}
} \\ 
\toprule
\multicolumn{1}{l}{} & Composite.frequency & Relative.frequency.per.1000.words & Commonality \\ 
\midrule\addlinespace[2.5pt]
的 & 856 & 47.84 & 1 \\ 
我 & 839 & 46.89 & 1 \\ 
個 & 642 & 35.88 & 1 \\ 
\bottomrule
\end{longtable}
\begin{longtable}{l|rrr}
\caption*{
{\large Preview of Core Words for 5-Year-Olds}
} \\ 
\toprule
\multicolumn{1}{l}{} & Composite.frequency & Relative.frequency.per.1000.words & Commonality \\ 
\midrule\addlinespace[2.5pt]
我 & 960 & 42.13 & 1 \\ 
的 & 910 & 39.94 & 1 \\ 
個 & 734 & 32.21 & 1 \\ 
\bottomrule
\end{longtable}
\begin{longtable}{l|rrr}
\caption*{
{\large Preview of Core Words for 6-Year-Olds}
} \\ 
\toprule
\multicolumn{1}{l}{} & Composite.frequency & Relative.frequency.per.1000.words & Commonality \\ 
\midrule\addlinespace[2.5pt]
我 & 857 & 47.37 & 0.97 \\ 
的 & 671 & 37.09 & 0.97 \\ 
個 & 541 & 29.90 & 0.95 \\ 
\bottomrule
\end{longtable}

The complete list of each of these charts can be found also in the excel
file ``core\_words\_processed.xlsx'' under the corresponding ``Core
Words for X-Year-Olds'' tab.

\hypertarget{relationship-between-commonality-and-frequency}{%
\section{Relationship Between Commonality and
Frequency}\label{relationship-between-commonality-and-frequency}}

Next, I explore the relationship between commonality and frequency. To
provide one possible summary of this relationship, a smoothing spline
was fit to the set of core words (as defined by the combined group).
This can be found below. As expected, we see a generally increasing
relationship. Graphically, we also see increased variability in the
relationship as the commonality increases.

\begin{center}\includegraphics{Aided_AAC_Report_files/figure-latex/unnamed-chunk-3-1} \end{center}

The plot above can be used to summarize the relationship graphically as
well as to predict the frequency of a core word given its commonality.

Smoothing splines require choosing a parameter that controls the
``wiggliness'' of the line. Here, I chose a parameter by choosing the
curve that graphically best fit the data without appearing to overfit.
However, the smoothing parameter can also be chosen using cross
validation and results in a ``wigglier'' line that using intuition
appears to me to suffer from over-fitting.

Other potential models are explored in the Appendix. Namely, an
exponential model is fit as well as a discontinuous piecewise function.
These models have the advantage of having a fitted curve with a
closed-form equation. However, they do not fit the shape of the data as
well as the smoothing spline displayed above.

Alternatively, a smoothed scatterplot is also presented in the Appendix.

\hypertarget{core-words-between-different-age-groups}{%
\section{Core Words Between Different Age
Groups}\label{core-words-between-different-age-groups}}

Next, we look to see if the commonality of possible core words change
between different age groups. To do this, we consider all words that are
determined to be a core word in the 4-year-old, 5-year-old, or
6-year-old group. Below is a preview of a chart which catalogs the
groups in which each word is classified as a core word. The complete
charts can be found in the excel file ``core\_words\_processed.xlsx''
under the ``Core Words Comparison'' tab.

\begin{longtable}{l|lllrrr}
\caption*{
{\large Preview of Core Words for Different Age Groups}
} \\ 
\toprule
\multicolumn{1}{l}{} & X4.Year.Olds & X5.Year.Olds & X6.Year.Olds & Composite.frequency & Relative.frequency.per.1000.words & Commonality \\ 
\midrule\addlinespace[2.5pt]
我 & x & x & x & 2656 & 45.19 & 0.99 \\ 
的 & x & x & x & 2437 & 41.46 & 0.99 \\ 
個 & x & x & x & 1917 & 32.62 & 0.98 \\ 
是 & x & x & x & 1507 & 25.64 & 0.98 \\ 
有 & x & x & x & 1466 & 24.94 & 0.98 \\ 
就 & x & x & x & 1165 & 19.82 & 0.96 \\ 
\bottomrule
\end{longtable}

In the above chart, if a word is considered core within a certain age
group, this is represented with an ``x'' in the corresponding cell. This
chart also includes the frequency and commonality in the combined group
and orders each word by decreasing frequency.

For each of the words considered core for any age group, we conduct a
statistical test to see if the commonality of the word significantly
differs between two different age groups. This is done though a z-test
for the difference of proportions (or equivalently a chi-squared test
for independence).

After accounting for multiple tests using the Bonferroni correction, we
find no significant difference between the commonality of any core word
in the age group of 4 verses the age group of 5. Similarly, we find no
significant difference between the age group of 5 and the age group of
6.

We do find a significant difference in the commonality of at least one
word between the 4-year-old age group and the 6-year-old age group.
Specifically, the word 她 is significantly more common in the 6-year-old
age group than in the 4-year-old age group with a commonality of 62\%
verses 13\%. This suggests that the core words between these two age
groups are significantly different. Thus, there may be an advantage to
creating different Aided AAC for 4-year-olds than for 6-year-olds if
considering the commonality threshold of 0.3 and frequency threshold of
0.5.

\hypertarget{appendix-other-models-of-the-relationship-between-commonality-and-frequency}{%
\section{Appendix: Other Models of the Relationship Between Commonality
and
Frequency}\label{appendix-other-models-of-the-relationship-between-commonality-and-frequency}}

We begin with an alternative scatterplot representation that smooths the
relationship by averaging the frequency of words with the same
commonality.

\begin{center}\includegraphics{Aided_AAC_Report_files/figure-latex/unnamed-chunk-6-1} \end{center}

In addition to the smoothing spline presented in the main report, we
also explored other models for the relationship between commonality and
frequency. Below are plots corresponding to two other potential models
followed by a brief discussion on each model.

We first consider an exponential model. One advantage to this model is
that the line of best fit can be expressed in a closed-form expression.
In fact, fitting an exponential curve to the relationship between
commonality and frequency we see can obtain the prediction

\[\text{Estimated Frequency} = 50.46\cdot 18.88^{c}\] for a word with
commonality \(c\). This curve is displayed below. Unfortunately, this
curve, while easy to interpret, does not fit the shape of the
relationship well.

\begin{center}\includegraphics{Aided_AAC_Report_files/figure-latex/unnamed-chunk-7-1} \end{center}

We next considered a piece-wise smooth model based on intuition. It is
possible that words can be described as either ``very common'' or ``not
very common''. The very common words can be thought of as the English
equivalent of ``I'', ``is'', etc. The very common words correspond to
the words on the far right-hand side of the scatterplots.

These very common words appear to have a different distribution than the
other words. For example, the variance in the very common words is much
higher. There is also much less of a clear pattern. One possible
hypothesis is that not very common words fit an exponential distribution
while very common words are randomly distributed around some common
average frequency.

To fit a model using the above description, we must determine (1) a
threshold for words being classified as very common, (2) an exponential
model for the not very common words, and (3) an average frequency for
very common words.

Below is the curve that minimizes squared error which uses a threshold
of 0.98 commonality. This model has the same benefit as the exponential
model in that the relationship can be expressed in closed-form, while
having a much closer fit to the true data.

\[\text{Estimated Frequency} = \begin{cases} 36.22\cdot 22.02^{c} & \text{if } c\leq .98,\\  1682.57 & \text{if } c> .98, \end{cases}\]
for a word with commonality \(c\).

\begin{center}\includegraphics{Aided_AAC_Report_files/figure-latex/unnamed-chunk-8-1} \end{center}

One potential drawback to the above curve is that the model performance
is very sensitive to the choice of threshold. Also, words that have
commonality either just to the left or the threshold or just to the
right of the threshold will have drastically different estimates of
frequency which is not ideal.

Because of the drawbacks of the above models, I believe the smoothing
spline given in the main report is a better choice for representing the
relationship between commality and frequency.

\end{document}
